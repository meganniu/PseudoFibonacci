Given the two initial terms of the sequence, the program displays the next 10 terms. Each terms of the sequence are calculated with the formula for the Fibonacci sequence\+: Fn = Fn-\/1 + Fn-\/2

\subsection*{Prerequisites}

In order to run the unit tests in the {\ttfamily \hyperlink{classTestFibonacci}{Test\+Fibonacci}} class, the libraries and headers for cppunit must be installed.

On Ubuntu 16.\+04, cppunit libraries and headers can be installed by typing the following into the terminal 
\begin{DoxyCode}
1 sudo apt-get install libcppunit-dev
\end{DoxyCode}
 The procedure for the installation of cppunit is similar on other Linux operating systems.

\subsection*{Compiling}

The executable files can be compiled using the makefile included. To create both the {\ttfamily Run\+Fibonacci} and the {\ttfamily \hyperlink{classTestFibonacci}{Test\+Fibonacci}} executables, navigate to the \hyperlink{classPseudoFibonacci}{Pseudo\+Fibonacci} folder and use


\begin{DoxyCode}
1 make all
\end{DoxyCode}
 To compile just the {\ttfamily \hyperlink{classPseudoFibonacci}{Pseudo\+Fibonacci}} class and create the {\ttfamily Run\+Fibonacci} executable, use 
\begin{DoxyCode}
1 make run
\end{DoxyCode}
 To compile the automated tests and create the {\ttfamily \hyperlink{classTestFibonacci}{Test\+Fibonacci}} executable, use 
\begin{DoxyCode}
1 make test
\end{DoxyCode}
 \subsection*{Running the program}

After compiling using {\ttfamily make all} or {\ttfamily make run}, run the program by using the following in terminal 
\begin{DoxyCode}
1 ./RunFibonacci
\end{DoxyCode}
 When prompted, type the first and second terms of the sequence. The program accepts integers and floating point numbers. For instance, 
\begin{DoxyCode}
1 First Number: -1.45
2 Second Number: 3
\end{DoxyCode}
 is considered valid. The program will not accept the input of non-\/numerical characters (exluding the decimal point).

\subsection*{Running the tests}

After compiling the tests in {\ttfamily Test\+Fibonacci.\+cpp} using {\ttfamily make all} or {\ttfamily make test}, run the tests using the following in terminal 
\begin{DoxyCode}
1 ./TestFibonacci
\end{DoxyCode}


\subsubsection*{Test Breakdown}


\begin{DoxyEnumerate}
\item {\ttfamily test\+Positives} tests that the output is correct when two positive rational numbers are entered.
\item {\ttfamily test\+Negatives} tests that the output is correct when two negative rational numbers are entered.
\item {\ttfamily test\+Mixed} tests that the ouput is correct when a postive and a negative rational number are entered.
\item {\ttfamily test\+Zeroes} tests that the output is correct when two zeroes are entered.
\item {\ttfamily test\+Length} tests that the length of the output is correct (i.\+e. 10 terms are calculated). It does not test for the correct values in the output.
\item {\ttfamily test\+Integers} tests that the ouput is correct when two integers are entered.
\item {\ttfamily test\+Display} tests that the sequence is correctly displayed in terms of values and formatting.
\end{DoxyEnumerate}

{\bfseries N\+O\+TE\+:} In the {\ttfamily \hyperlink{classTestFibonacci}{Test\+Fibonacci}} class, the {\ttfamily equal} method is used to compare the floating-\/point values to determine if the content of two containers are the same. Due to the inaccuracy of floating-\/point values, an error of {\ttfamily 0.\+0000000000001} has been allowed. This is the smallest value of error, as determined through trial and error. (Although the process seems a tad shady and the acceptable error may need adjusting when comparing smaller values.)

\subsection*{Documentation}

Documentation for all the classes can be acessed using the makefile. For firefox, type


\begin{DoxyCode}
1 openDocFirefox
\end{DoxyCode}
 For google chrome, type 
\begin{DoxyCode}
1 openDocChrome
\end{DoxyCode}
 If your preferred browser is not Firefox or Chrome, type 
\begin{DoxyCode}
1 BROWSER html/index.html
\end{DoxyCode}
 and subsitute {\ttfamily B\+R\+O\+W\+S\+ER} for your preferred browser.

\subsection*{Author}

\href{https://github.com/meganniu}{\tt Megan Niu}

\subsection*{Acknowledgments}

The {\ttfamily main} function of {\ttfamily Test\+Fibonacci.\+cpp} was adjusted from a cppunit tutorial found \href{yolinux.com/TUTORIALS/CppUnit.html}{\tt here}. 2-\/clause B\+SD License\+: \href{http://www.yolinux.com/YoLinux-Terms.html}{\tt http\+://www.\+yolinux.\+com/\+Yo\+Linux-\/\+Terms.\+html} 